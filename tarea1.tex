\section{Tarea 1}
Bird et all 17.A
\begin{itemize}
\item[1.-] Predicción binaria a baja densidad. 
    Estimar $D_{AB}$  para el sistema metano-etano a 293°K y 1 atm por medio de los siguientes métodos.
    \begin{itemize}
        \item [a)-] Ecuación 1.53
        \item [b)-] Ecuación 1.53 y gráfica Fig 1.51 utilizando las T y P reducidas $T_{r}=\frac{T}{\sqrt{T_{CA}T_{CB}}}$ , $P_{r}=\frac{P}{\sqrt{P_{CA}P_{CB}}}$
        \item [c)-] Ecuación 1.46, 1.48 y 1.49, y los parámetros de Lennard-Joues del Apéndice C.
        \item [d)-] Ecuación 1.46 en los parámetros de Lennard-Jones estimados a partir de las propiedades críticas siguientes:
         $$\frac{\epsilon}{k_{13}}=0.77\sqrt{T_{CA}T_{CB}}  ,  T=\frac{2.44}{2} \left[ \left(\frac{T_{CA}}{P_{CA}}\right)^{1/5}+\left(\frac{T_{CB}}{P_{CB}}\right)^{1/3} \right] $$
    \end{itemize} 
    Respuestas ($cm^2/s$) : a)- 0.152; b)- 0.138; c)- 0.146; d)- 0.138
\item[2.-] Autodifusión de mercurio líquido Bird et all 17 A3.
La difusividad del $Hg^{203}$ en mercurio líquido normal se ha medido con datos de viscosidad y volumen másico. Comparar los datos experimentales con aquellos obtenidos 
por la ec. 160.
    \begin{table}[H]  %Aquí empieza la tabla%
    \centering  %Le digo  que se centre %
    \begin{tabular}{cccc}
    \hline
    \textbf{T(K)} & \textbf{$ D_{AB} $ $cm^2/s$} & \textbf{$\mu$ cP} & \textbf{$\hat{V}$ $cm^3/s$} \\ \hline
            275.7 & $1.52 \times 10^{-5}$ & 1.68 & 0.0736 \\ 
            289.6 & $1.68 \times 10^{-5}$ & 1.56 & 0.0737 \\ 
            364.2 & $2.57 \times 10^{-5}$ & 1.27 & 0.0748 \\ \hline
        \end{tabular}
    \end{table}  

Bird et all 17.A.5    
\item[3.-] Cálculo de la difusividad de una muestra binaria a alta densidad. 
Predecir $pD_{AB}$ para una mezcla equimolar de $N_{2}$ y $C_{2}H_{6}$ a 288.2 K y 40 atm.
    \begin{itemize}
     \item [a)-] Usar el valor de $D_{AB}$ a 1 atm de  0.148 $cm^2/s$ a T=298.2 Ky la gráfica de la Fig.1.51
     \item [b)-] Usar la ecuación 1.55 y la Fig 1.51
    \end{itemize}    
Respuesta a).- $5.8*10^{-6}$ gmol/cms; b).- $5.3*10^{-6}$ gmol/cms
\item[4.-] Prob. 17.A.6 Bird
Difusividad y número de Schmidit para mezclas cloro-aire. 
    \begin{itemize}
    \item [a)-] Predecir $D_{AB}$ para mezclas cloro-aire a 75°F y 1 atm.
    Utilizar los parámetros del Apéndice C.
    \item [b)-] Calcular (a) utilizando la ec. 1.53
    \item [c)-]Utilizar los resultados de (a) para estimar los valores del número de Schmidit para mezclas cloro-aire a 297 K y 1 atm para las siguientes fracciones mol y viscosidades:
    0, $1.83*10^{-4} poises$; 0.25, $1.64*10^{-4} poises$; 0.5, $1.5*10^{-4} poises$; 0.75, $1.39x10^{-4} poises$; 1, $1.31*10^{-4} poises$
    $$ pD_{AB}= \frac{\rho}{RT}D_{AB}; \space  Sc=\frac{\mu}{M_{C}D_{AB}}=\frac{\mu}{(x_{A}M_{A}+x_{B}M_{B})pD_{AB}}$$
\end{itemize}
Respuestas: a).- 0.121 $cm/s$; b).- 0.124 $cm/s$; c).- Sc=1.27, 0.832, 0.602, 0.463, 0.372
\item[5.-] Probl. 17.A.8 Bird.
Corrección para la difusividad a altas densidades. 
El valor medido para $pD_{AB}$ de una mezcla de 80\% mol de $CH_{4}$ Y 20\% mol de $C_{2}H_{6}$ a 313 K y 136 atm es $6x10^-6$ gmol/cms.
Calcular $pD_{AB}$ para esa mezcla a 136 atm y 351 K usando la Fig 1.51.

Respuesta $6.3*10^{-6} gmol/cm s$. Observado $6.33*10^{-6 }gmol/cm s$
\item[6.-]   Probl 17 A.10 Bird 
Cálculo de difusividad de líquidos 
\begin{itemize}
     \item [a)-] Calcular la difusividad de una solución diluida de ácido acético a 12.5°C utilizando la ec. 1.62. La densidad del ácido acético es 0.937 $g/cm^3$ en el punto de ebullición.
     \item [b)-] La difusividad de una solución diluida de metanol a 15°C es $1.28*10^{-5} cm^2/s$. Calcular la difusividad de esa solución a 100°C. 
     Las viscosidades a 15°C y 100°C son 1.14 cp y 0.28cp.
     La viscosidad de la solución diluida es 1.22 cp.
    \end{itemize}
Respuesta (b).- $6.7*10^{-5} cm^2/s$.
\end{itemize}
\newpage


 